\documentclass[a4paper]{article}

\usepackage{expl3}
\usepackage[overload]{empheq}
\usepackage{breqn}

\begin{document}

This is a test document for general fontification support of macros
and environments provided by \LaTeX{} and other packages within
AUC\TeX.  As it is not part of \verb|ert|-tests, it does not have an
\verb|out|-counterpart.

\part{Macros}

\section{Special characters}

\subsection{Standard reserved characters}

\subsubsection{\protect\LaTeX{} input}

The following symbols are reserved characters that have a special
meaning under \LaTeX{}.
\begin{center}
  \begin{tabular}{@{}*{8}{c}}
    \hline
    \verb|#|   & \verb|$| & \verb|%| & \verb|^|
    & \verb|&| & \verb|_| & \verb|{| & \verb|}| \\
    \verb|~|   & \verb|`| & \verb|'| & \verb|=|
    & \verb|.| \\
    \hline
  \end{tabular}\\
\end{center}
In a document, characters in the first line are printed by using a
prefix \verb|\| (backslash).  The characters in the second line become
control symbols taking an argument when prefixed with a backslash.

\subsubsection{AUC\protect\TeX{} fontification}

AUC\TeX{} has the following strategy for fontification:
\begin{description}
\item[Control symbols without argument] do not get any fontification
  as they only print a special character and have a textual context,
  e.g. \$10, 5\%, Mr.\&Mrs.  They are listed below:
  \begin{center}
    \begin{tabular}{@{}*{4}{cl}@{}}
      \hline
      \#   & \verb|\#| & \$ & \verb|\$|
      & \% & \verb|\%| & \& & \verb|\&| \\
      \_   & \verb|\_| & \{ & \verb|\{|
      & \} & \verb|\}|                  \\
      \hline
    \end{tabular}
\end{center}
\item[Control symbols with argument] do not receive any fontification.
  These macros take a mandatory argument, but they have a textual
  context.  Since the argument is usually not enclosed in braces,
  fontification would be rather distracting:
  \begin{center}
    \begin{tabular}{@{}*{4}{cl}@{}}
      \hline
      \`o   & \verb|\`o| & \'o & \verb|\'o|
      & \^o & \verb|\^o| & \~o & \verb|\~o|  \\
      \=o   & \verb|\=o| & \.o & \verb|\.o|
      & \"o & \verb|\"o|                     \\
      \hline
    \end{tabular} \\[1ex]
    \begin{minipage}{0.4\linewidth}\centering
      H\^{o}tel, na\"\i ve, \'{e}l\`{e}ve,\\
      sm\o rrebr\o d, !`Se\~{n}orita!
    \end{minipage}\quad
    \begin{minipage}{0.4\linewidth}\centering
      H\^otel, na\"\i ve, \'el\`eve,\\
      sm\o rrebr\o d, !`Se\~norita!
    \end{minipage}
  \end{center}
\item[Control words] receive fontification, e.g.
  \begin{center}
    \begin{tabular}{@{}*{4}{cl}@{}}
      \hline
      \u o    & \verb|\u o| & \v o & \verb|\v o|
      & \H o  & \verb|\H o| & \c o & \verb|\c o| \\
      \d o    & \verb|\d o| & \b o & \verb|\b o|
      & \t oo & \verb|\t oo|                     \\[6pt]
      \oe     & \verb|\oe|  & \OE  & \verb|\OE|
      & \ae   & \verb|\ae|  & \AE  & \verb|\AE|  \\
      \aa     & \verb|\aa|  & \AA  & \verb|\AA|
      & \c c  & \verb|\c c|                      \\[6pt]
      \o      & \verb|\o|   & \O   & \verb|\O|
      & \l    & \verb|\l|   & \L   & \verb|\L|   \\
      \i      & \verb|\i|   & \j   & \verb|\J|
      & \ss   & \verb|\ss|                       \\
      \hline
    \end{tabular}\\[1ex]
  \end{center}
\end{description}

\subsection{@ character}

\subsubsection{\protect\LaTeX{} input}

In regular text, \verb|@| is not a special character and can be used
as `foo@bar'.  Further, \verb|\@| can be used to force a wide space
after an uppercase character, e.g. PC\@.

On the other hand, \verb|@| is used as ``letter'' for defining internal
macros, e.g. \verb|\@gobble|.

\subsubsection{AUC\protect\TeX{} fontification}

Being a non-textual macro, \verb|\@| receives a fontification in
AUC\TeX.  Used as letter as part of a macro, it gets the fontification
as the rest. Example \bgroup\ttfamily \string\@gobble\egroup.

\subsection{\_ and : characters}

\subsubsection{\protect\LaTeX{} input}

For \LaTeXe, the behavior of \verb|_| and \verb|\_| was described
above.  \verb|:| is not special for regular text.  \verb|\:| is a
spacing macro ($\frac{4}{18}$ quad) within math mode.

\LaTeX3 does not use \verb|@| as ``letter'' for defining internal
macros.  Instead, the symbols \verb|_| and \verb|:| are used in
internal macro names to provide structure.  These extra letters are
used only between parts of a macro name (no strange vowel
replacement)\cite{expl3}.

\subsubsection{AUC\protect\TeX{} fontification}

\verb|\:| is only allowed in math mode; fontification is done there,
e.g. $a\:\:+b$ or
\begin{equation}
  \int_1^2 \ln x \mathrm{d}x
  \qquad
  \int_1^2 \ln x \:\:\:\mathrm{d}x
\end{equation}

Regarding \LaTeX3, \verb|expl3| says:
\begin{quote}
  3.2.1 Separating private and public material \\
  Functions created by a module may either be ``public'' (documented
  with a defined interface) or ``private'' (to be used only within
  that module, and thus not formally documented).  It is important
  that only documented interfaces are used; [...] \\
  To allow clear separation of these two cases, the following
  convention is used. Private functions should be defined with
  \verb|__| added to the beginning of the module name. Thus
  \begin{quote}
    \ttfamily \catcode`\_11\relax
    \string\module_foo:nnn
  \end{quote}
  is a public function which should be documented while
  \begin{quote}
    \ttfamily \catcode`\_11\relax
    \string\__module_foo:nnn
  \end{quote}
  is private to the module, and should not be used outside of that
  module.
\end{quote}
%
Hopefully, this means that \verb|\__module_foo:nnn| macros will not be
used somewhere in the preamble of a \verb|.tex| file -- they should
appear only in a \verb|.dtx| file.

AUC\TeX{} provides a style file \verb|expl3.el| containing this code:
\begin{verbatim}
(defvar LaTeX-expl3-syntax-table
  (let ((st (copy-syntax-table LaTeX-mode-syntax-table)))
    ;; Make _ and : symbol chars
    (modify-syntax-entry ?\_ "_" st)
    (modify-syntax-entry ?\: "_" st)
    st))
\end{verbatim}
It changes the syntax for \verb|_| and \verb|:| from
\textsl{punctuation} to \textsl{symbol}.  For public functions
mentioned above, this results in correct fontification in regular
\verb|.tex| files.  For private functions, \verb|font-latex.el|
provides some code for doc\TeX{} mode to fontify them correctly.

\section{Math mode}

\subsection{\protect\LaTeX{} input}

In-line math is typeset with plain \TeX{}
\verb|$|\,\textsl{formula\_text}\,\verb|$| or \LaTeX{} shorthand
\verb|\(|\,\textsl{formula\_text}\,\verb|\)|, e.g. $(a+b)^2$ is equal
to \(a^2+2ab+b^2\).

\subsection{AUC\protect\TeX{} fontification}

AUC\TeX{} fontifies math with \texttt{font-latex-math-face}.  There is
a bug report \#26630 for this issue: {\bfseries Text before, $(a+b)^2$
  is equal to \(a^2+2ab+b^2\)!}  First math expression is fontified
with
\begin{quote}
\verb|(font-latex-math-face font-latex-bold-face)|
\end{quote}
while the second with
\begin{quote}
\verb|(font-latex-bold-face font-latex-math-face)|
\end{quote}

\part{Environments}

\section{Math mode}

\subsection{\protect\LaTeX{} input}

Standard \LaTeX{} math environments are \verb|equation|,
\verb|displaymath| and others.  \verb|amsmath| package provides
environments like \verb|align|, \verb|flalign| etc.  These
environments do not take any arguments.  Environments like
\verb|alignat|, \verb|xalignat| and \verb|xxalignat| take a mandatory
argument.  Other math environments provided by packages like
\verb|empheq.sty| or \verb|breqn.sty| take an optional and/or
mandatory argument.

\subsection{AUC\protect\TeX{} fontification}

AUC\TeX{} fontifies the entire math content with
\texttt{font-latex-math-face}.  The optional and mandatory argument(s)
should not be fontified.  Spaces or line breaks are used in order to
distinguish argument from math content, i.e.
\begin{quote}
\verb|\begin{<mathenv>}[<opt-arg>]{<mand-arg>}|
\end{quote}
will be fontified differently than
\begin{quote}
\verb|\begin{<mathenv>}[<opt-arg>] {<math-content>}|
\end{quote}
or
\begin{quote}
\verb|\begin{<mathenv>} [<math-content>]{<math-content>}|
\end{quote}

The relevant functions is \verb|font-latex.el| are
\begin{quote}
\verb|font-latex-match-math-envII|\quad and \\
\verb|font-latex-extend-region-backwards-math-envII|
\end{quote}

\subsubsection{Standard \protect\LaTeX{}}

Examples taken from \cite{voss16}.
\begin{equation}
f(x)=\prod_{i=1}^{n}\left(i-\frac{1}{2i}\right)
\end{equation}
or
\begin{displaymath}
f(x)=\prod_{i=1}^{n}\left(i-\frac{1}{2i}\right)
\end{displaymath}
or
\[ f(x)=\prod_{i=1}^{n}\left(i-\frac{1}{2i}\right) \]
or
\begin{eqnarray*}
  y & = & d\\
  y & = & cx+d\\
  y & = & bx^{2}+cx+d\\
  y & = & ax^{3}+bx^{2}+cx+d
\end{eqnarray*}

\subsubsection{\texttt{amsmath} package}

Examples taken from \cite{voss16}.
\begin{align}
  y &= d & z &= 1 \\
  y &= cx+d & z &= x+1 \\
  y_{12} &= bx^{2}+cx+d & z &= x^{2}+x+1\nonumber \\
  y(x) &= ax^{3}+bx^{2}+cx+d & z &= x^{3}+x^{2}+x+1
\end{align}
or
\begin{alignat}{2}
  y &= d & z &= 1 \\
  y &= cx+d & z & =x+1 \\
  y_{12} &= bx^{2}+cx+d & z &= x^{2}+x+1\nonumber \\
  y(x) &= ax^{3}+bx^{2}+cx+d & z &= x^{3}+x^{2}+x+1
\end{alignat}

\subsubsection{\texttt{breqn} package}

Examples taken from \cite{breqn}:
\begin{dmath}
  f(x)=\frac{1}{x} \condition{for $x\neq 0$}
\end{dmath}
or
\begin{dmath}[label={sna74}]
  \frac{1}{6} \left(\sigma(k,h,0) +\frac{3(h-1)}{h}\right)
  +\frac{1}{6} \left(\sigma(h,k,0) +\frac{3(k-1)}{k}\right)
  =\frac{1}{6} \left(\frac{h}{k} +\frac{k}{h} +\frac{1}{hk}\right)
  +\frac{1}{2} -\frac{1}{2h} -\frac{1}{2k},
\end{dmath}
or \newcommand\mx[1]{\begin{math}#1\end{math}}%
\begin{dseries}[frame]
  \mx{v^{[2]} =(0,5,5,0,9,5,1,0)},
  \mx{v^{[3]} =(0,9,11,9,10,12,0,1)}.
\end{dseries}

\begin{dgroup*}
  \begin{dmath*}
    H_1^3 = x_1 + x_2 + x_3
  \end{dmath*},
  \begin{dmath*}
    H_2^2 = x_1^2 + x_1 x_2 + x_2^2 - q_1 - q_2
  \end{dmath*},
  \begin{dsuspend}
    and
  \end{dsuspend}
  \begin{dmath*}
    H_3^1 = x_1^3 - 2x_1 q_1 - x_2 q_1
  \end{dmath*}.
\end{dgroup*}

\subsubsection{\texttt{empheq} package}

Examples taken from \cite{empheq}:
\begin{empheq}{align*}
  a&= b \tag{*}\\
  E&= mc^2 + \int_a^a x\,dx
\end{empheq}
or
\begin{empheq}{alignat=2}
  a &= b &\quad c &= d \\
  \text{this} &= \text{that} &\quad \mathit{fish}&\neq fish
\end{empheq}
or
\begin{empheq}[innerbox=\fbox,
  left=L\Rightarrow]{align}
  a&=b\\
  E&=mc^2 + \int_a^a x\, dx
\end{empheq}
or
\begin{empheq}[
  left={\parbox[c][\EmphEqdisplayheight+\EmphEqdisplaydepth][t]
    {4.5cm}
    {You may find this kind of description useful.}\enspace}%
  ]{align}
  a&=\int_0^1 x\,dx +\frac{foo + bar}{baz}\\
  E&= mc^2
\end{empheq}

\begin{thebibliography}{9}\raggedright
\bibitem{breqn} Downes, Michael J.; H\o gholm, Morten: The
  \texttt{breqn} package.  Released 2017/01/27.
\bibitem{empheq} H\o gholm, Morten; Madsen, Lars: The \texttt{empheq}
  package.  Released 2014/08/04.
\bibitem{expl3} The \LaTeX3 Project: The expl3 package and \LaTeX3
  programming.  Released 2017/04/01.
\bibitem{voss16} Vo\ss, Herbert: Mathematical Typesetting with
  \LaTeX.  TUG-Version 0.32, released 2016/11/08.
\end{thebibliography}

\end{document}

%%% Local Variables:
%%% mode: latex
%%% TeX-master: t
%%% End:
